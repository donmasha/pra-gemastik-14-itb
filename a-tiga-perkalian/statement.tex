\documentclass{article}

\usepackage{geometry}
\usepackage{amsmath}
\usepackage{graphicx, eso-pic}
\usepackage{listings}
\usepackage{hyperref}
\usepackage{multicol}
\usepackage{fancyhdr}
\usepackage{mathtools}

\DeclarePairedDelimiter\floor{\lfloor}{\rfloor}

\pagestyle{fancy}
\fancyhf{}
\hypersetup{ colorlinks=true, linkcolor=black, filecolor=magenta, urlcolor=cyan}
\geometry{ a4paper, total={170mm,257mm}, top=20mm, right=20mm, bottom=20mm, left=20mm}
\lhead{Pra GEMASTIK 14 ITB | Competitive Programming}
\setlength{\parindent}{0pt}
\setlength{\parskip}{0.3em}
\renewcommand{\headrulewidth}{0pt}
\rfoot{\thepage}
\lfoot{Pra GEMASTIK 14 ITB | Competitive Programming}
\lstset{
    basicstyle=\ttfamily\small,
    columns=fixed,
    extendedchars=true,
    breaklines=true,
    tabsize=2,
    prebreak=\raisebox{0ex}[0ex][0ex]{\ensuremath{\hookleftarrow}},
    frame=none,
    showtabs=false,
    showspaces=false,
    showstringspaces=false,
    prebreak={},
    keywordstyle=\color[rgb]{0.627,0.126,0.941},
    commentstyle=\color[rgb]{0.133,0.545,0.133},
    stringstyle=\color[rgb]{01,0,0},
    captionpos=t,
    escapeinside={(\%}{\%)}
}

\begin{document}

\begin{center}
    \section*{Problem A - Tiga Perkalian} % ganti judul soal
    \begin{tabular}{ | c c | }
        \hline
        Batas Waktu  & 1s \\    % jangan lupa ganti time limit
        Batas Memori & 64MB \\  % jangan lupa ganti memory limit
        \hline
    \end{tabular}
\end{center}

\subsection*{Deskripsi}

Diberikan $N$ bilangan bulat, $A_1, A_2, \dots, A_N$, Anda disuruh mengambil 3 bilangan bulat dari $N$ bilangan bulat tersebut, dan mencari \textbf{Perkalian terbesar} dan \textbf{Perkalian terkecil} yang dapat diperoleh!

\subsection*{Format Masukan}

Baris pertama terdiri dari satu bilangan bulat positif $N$ ($1 \leq N \leq 10^5$), menyatakan banyaknya bilangan.

Baris kedua, berisi $N$ bilangan bulat $A_1, A_2, \dots, A_N$ $(-10^6 \leq A_i \leq 10^6)$, menyatakan nilai dari $N$ bilangan bulat yang dipisahkan dengan spasi.

\subsection*{Format Keluaran}

Keluarkan dua bilangan berupa perkalian terbesar dan perkalian terkecil yang dapat diperoleh dengan mengambil 3 bilangan, dipisahkan dengan spasi

\begin{multicols}{2}
\subsection*{Contoh Masukan}
\begin{lstlisting}
6
-4 9 10 13 -9 2 
\end{lstlisting}
\columnbreak
\subsection*{Contoh Keluaran}
\begin{lstlisting}
1170 -1170
\end{lstlisting}
\vfill
\null
\end{multicols}

\subsection*{Penjelasan}

Dengan mengambil tiga bilangan, maka perkalian terbesar diperoleh dari $A_2 \times A_3 \times A_4$ dan perkalian terkecil diperoleh dari $A_3 \times A_4 \times A_5$.

\pagebreak

\end{document}