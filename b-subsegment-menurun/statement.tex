\documentclass{article}

\usepackage{geometry}
\usepackage{amsmath}
\usepackage{graphicx, eso-pic}
\usepackage{listings}
\usepackage{hyperref}
\usepackage{multicol}
\usepackage{fancyhdr}
\usepackage{mathtools}

\DeclarePairedDelimiter\floor{\lfloor}{\rfloor}

\pagestyle{fancy}
\fancyhf{}
\hypersetup{ colorlinks=true, linkcolor=black, filecolor=magenta, urlcolor=cyan}
\geometry{ a4paper, total={170mm,257mm}, top=20mm, right=20mm, bottom=20mm, left=20mm}
\lhead{Pra GEMASTIK 14 ITB | Competitive Programming}
\setlength{\parindent}{0pt}
\setlength{\parskip}{0.3em}
\renewcommand{\headrulewidth}{0pt}
\rfoot{\thepage}
\lfoot{Pra GEMASTIK 14 ITB | Competitive Programming}
\lstset{
    basicstyle=\ttfamily\small,
    columns=fixed,
    extendedchars=true,
    breaklines=true,
    tabsize=2,
    prebreak=\raisebox{0ex}[0ex][0ex]{\ensuremath{\hookleftarrow}},
    frame=none,
    showtabs=false,
    showspaces=false,
    showstringspaces=false,
    prebreak={},
    keywordstyle=\color[rgb]{0.627,0.126,0.941},
    commentstyle=\color[rgb]{0.133,0.545,0.133},
    stringstyle=\color[rgb]{01,0,0},
    captionpos=t,
    escapeinside={(\%}{\%)}
}

\begin{document}

\begin{center}
    \section*{Problem B - Subsegment Menurun} % ganti judul soal
    \begin{tabular}{ | c c | }
        \hline
        Batas Waktu  & 1s \\    % jangan lupa ganti time limit
        Batas Memori & 128MB \\  % jangan lupa ganti memory limit
        \hline
    \end{tabular}
\end{center}

\subsection*{Deskripsi}

Diberikan $N$ bilangan bulat, $A_1, A_2, \dots, A_N$, sebut saja sebagai barisan $A$.

Didefinisikan barisan $A_i, A_{i+1}, \dots  , A_j$ $(1 \leq i \leq j \leq n)$ sebagai subsegmen dari barisan $A$.

Anda dapat mengubah maksimal satu bilangan menjadi bilangan apapun (bisa tidak ada yang diubah). Lalu, Anda disuruh mencari panjang dari subsegment terpanjang yang barisan-nya menurun! $(A_i > A_{i + 1} > \dots > A_j)$.

\subsection*{Format Masukan}

Baris pertama terdiri dari satu bilangan bulat positif $N$ ($1 \leq N \leq 10^5$), menyatakan banyaknya bilangan.

Baris kedua, berisi $N$ bilangan bulat $A_1, A_2, \dots, A_N$ $(0 \leq A_i \leq 10^9)$, menyatakan nilai dari $N$ bilangan bulat yang dipisahkan dengan spasi.

\subsection*{Format Keluaran}

Keluarkan satu bilangan bulat berupa jawaban dari persoalan ini.

\begin{multicols}{2}
\subsection*{Contoh Masukan}
\begin{lstlisting}
8
1 3 5 1 8 0 1 2
\end{lstlisting}
\columnbreak
\subsection*{Contoh Keluaran}
\begin{lstlisting}
3
\end{lstlisting}
\vfill
\null
\end{multicols}

\subsection*{Penjelasan}

Salah satu cara mendapatkan jawaban adalah dengan mengubah $A_6 = 2$ sehingga subsegment yang terpanjang yang menurun adalah $A_5, A_6, A_7$.

\pagebreak

\end{document}