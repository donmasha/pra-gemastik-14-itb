\documentclass{article}

\usepackage{geometry}
\usepackage{amsmath}
\usepackage{graphicx, eso-pic}
\usepackage{listings}
\usepackage{hyperref}
\usepackage{multicol}
\usepackage{fancyhdr}
\usepackage{mathtools}

\DeclarePairedDelimiter\floor{\lfloor}{\rfloor}

\pagestyle{fancy}
\fancyhf{}
\hypersetup{ colorlinks=true, linkcolor=black, filecolor=magenta, urlcolor=cyan}
\geometry{ a4paper, total={170mm,257mm}, top=20mm, right=20mm, bottom=20mm, left=20mm}
\lhead{Pra GEMASTIK 14 ITB | Competitive Programming}
\setlength{\parindent}{0pt}
\setlength{\parskip}{0.3em}
\renewcommand{\headrulewidth}{0pt}
\rfoot{\thepage}
\lfoot{Pra GEMASTIK 14 ITB | Competitive Programming}
\lstset{
    basicstyle=\ttfamily\small,
    columns=fixed,
    extendedchars=true,
    breaklines=true,
    tabsize=2,
    prebreak=\raisebox{0ex}[0ex][0ex]{\ensuremath{\hookleftarrow}},
    frame=none,
    showtabs=false,
    showspaces=false,
    showstringspaces=false,
    prebreak={},
    keywordstyle=\color[rgb]{0.627,0.126,0.941},
    commentstyle=\color[rgb]{0.133,0.545,0.133},
    stringstyle=\color[rgb]{01,0,0},
    captionpos=t,
    escapeinside={(\%}{\%)}
}

\begin{document}

\begin{center}
    \section*{Problem C - Hukuman Telat} % ganti judul soal
    \begin{tabular}{ | c c | }
        \hline
        Batas Waktu  & 2s \\    % jangan lupa ganti time limit
        Batas Memori & 128MB \\  % jangan lupa ganti memory limit
        \hline
    \end{tabular}
\end{center}

\subsection*{Deskripsi}

Pada tim pemrograman ITB. Ada $P$ orang dalam tim tersebut, dinomori 1 sampai $P$, dan setiap kali orang terlambat, ditulis nomor mereka di papan tulis. Melihat papan tulis, ternyata terjadi $N$ kali terlambat yang dicatat secara total. Pelatih Mastre telah memutuskan untuk membuat para orang melakukan push-up sebagai hukuman. Orang ke-$K$ harus menghitung berapa kali total orang nomor $1$ sampai $K$ terlambat, kemudian melakukan push-up sebanyak itu. Donbasta bertanya-tanya apakah Pelatih Mastre terlalu kejam kepada para orangnya, jadi Donbasta bertanya kepadanya, "Berapa banyak orang yang paling banyak melakukan $X$ push-up?". Donbasta sangat pelupa, jadi dia mungkin bertanya kepada Pelatih Mastre hingga $Q$ pernyataan. Dia mungkin mengajukan beberapa pertanyaan dengan $X$ yang sama. Karena $Q$ dan $X$ mungkin besar, bantu Pelatih Mastre menjawab pertanyaan-pertanyaan ini!

\subsection*{Format Masukan}

Baris pertama berisi dua bilangan bulat yang dipisahkan spasi, $P$ dan $N$ $(1 \leq P \leq 10^9,  1 \leq N \leq 10^5)$.

Baris kedua berisi $N$ bilangan bulat yang dipisahkan oleh spasi: angka-angka di papan tulis.

Baris ketiga berisi bilangan bulat $Q$, $(1 \leq Q \leq 10^5)$ jumlah pertanyaan yang akan diajukan Donbasta.

$Q$ baris berikutnya mewakili pertanyaan. Setiap pertanyaan berisi bilangan bulat $X$  $(0 \leq X \leq 10^9)$.

\subsection*{Format Keluaran}

Untuk setiap pertanyaan, keluarkan jawaban dari pertanyaan tersebut.

\begin{multicols}{2}
\subsection*{Contoh Masukan}
\begin{lstlisting}
4 6
4 2 1 2 1 2
5
0
1
2
3
4
\end{lstlisting}
\columnbreak
\subsection*{Contoh Keluaran}
\begin{lstlisting}
0
0
1
1
1
\end{lstlisting}
\vfill
\null
\end{multicols}

\pagebreak

\end{document}