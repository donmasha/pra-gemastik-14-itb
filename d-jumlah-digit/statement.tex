\documentclass{article}

\usepackage{geometry}
\usepackage{amsmath}
\usepackage{graphicx, eso-pic}
\usepackage{listings}
\usepackage{hyperref}
\usepackage{multicol}
\usepackage{fancyhdr}
\usepackage{mathtools}

\DeclarePairedDelimiter\floor{\lfloor}{\rfloor}

\pagestyle{fancy}
\fancyhf{}
\hypersetup{ colorlinks=true, linkcolor=black, filecolor=magenta, urlcolor=cyan}
\geometry{ a4paper, total={170mm,257mm}, top=20mm, right=20mm, bottom=20mm, left=20mm}
\lhead{Pra GEMASTIK 14 ITB | Competitive Programming}
\setlength{\parindent}{0pt}
\setlength{\parskip}{0.3em}
\renewcommand{\headrulewidth}{0pt}
\rfoot{\thepage}
\lfoot{Pra GEMASTIK 14 ITB | Competitive Programming}
\lstset{
    basicstyle=\ttfamily\small,
    columns=fixed,
    extendedchars=true,
    breaklines=true,
    tabsize=2,
    prebreak=\raisebox{0ex}[0ex][0ex]{\ensuremath{\hookleftarrow}},
    frame=none,
    showtabs=false,
    showspaces=false,
    showstringspaces=false,
    prebreak={},
    keywordstyle=\color[rgb]{0.627,0.126,0.941},
    commentstyle=\color[rgb]{0.133,0.545,0.133},
    stringstyle=\color[rgb]{01,0,0},
    captionpos=t,
    escapeinside={(\%}{\%)}
}

\begin{document}

\begin{center}
    \section*{Problem D - Jumlah Digit} % ganti judul soal
    \begin{tabular}{ | c c | }
        \hline
        Batas Waktu  & 2s \\    % jangan lupa ganti time limit
        Batas Memori & 512MB \\  % jangan lupa ganti memory limit
        \hline
    \end{tabular}
\end{center}

\subsection*{Deskripsi}

Jika $X$ merupakan bilangan bulat, maka $S(X)$ didefinisikan sebagai berikut:

\begin{center}
    $S(X) = $ jumlah digit dari $X$ (contoh: $S(135) = 1 + 3 + 5 = 9$)
\end{center}

Anda diberikan dua bilangan bulat $L$ dan $R$ $(L \leq R)$. Tugas Anda adalah menghitung:

$$S(L) + S(L + 1) + \dots + S(R - 1) + S(R)$$

Namun, Anda hanya perlu menghitung $S(X)$ jika bilangan $X$ berisi paling sedikit $P$ angka yang berbeda dan paling banyak $Q$ angka yang berbeda $(P \leq Q)$. Kemudian cetak hasil ini modulo $10^9 + 7$.

\textbf{Contoh}: Jika $P = Q = 1$ maka Anda harus menghitung semua bilangan dari $L$ sampai $R$ sedemikian rupa sehingga setiap bilangan dibentuk hanya dengan menggunakan satu angka.

Sehingga untuk $L = 10$, $R = 50$ jawabannya adalah $S(11) + S(22) + S(33) + S(44) = 20$.

\subsection*{Format Masukan}

Baris pertama terdiri dari dua bilangan bulat positif $L$, $R$ ($1 \leq L, R \leq 10^{18}$).

Baris kedua terdiri dari dua bilangan bulat positif $P$, $Q$ ($1 \leq P \leq Q \leq 19$).

\subsection*{Format Keluaran}

Keluarkan satu bilangan bulat berupa jawaban dari persoalan ini.

\begin{multicols}{2}
\subsection*{Contoh Masukan 1}
\begin{lstlisting}
10 50
1 1
\end{lstlisting}
\columnbreak
\subsection*{Contoh Keluaran 1}
\begin{lstlisting}
20
\end{lstlisting}
\vfill
\null
\end{multicols}

\begin{multicols}{2}
\subsection*{Contoh Masukan 2}
\begin{lstlisting}
1 10
1 2
\end{lstlisting}
\columnbreak
\subsection*{Contoh Keluaran 2}
\begin{lstlisting}
46
\end{lstlisting}
\vfill
\null
\end{multicols}

\begin{multicols}{2}
\subsection*{Contoh Masukan 3}
\begin{lstlisting}
13518012 16518012
1 2
\end{lstlisting}
\columnbreak
\subsection*{Contoh Keluaran 3}
\begin{lstlisting}
3632
\end{lstlisting}
\vfill
\null
\end{multicols}

\pagebreak

\end{document}