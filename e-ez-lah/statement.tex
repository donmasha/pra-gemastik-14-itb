\documentclass{article}

\usepackage{geometry}
\usepackage{amsmath}
\usepackage{graphicx, eso-pic}
\usepackage{listings}
\usepackage{hyperref}
\usepackage{multicol}
\usepackage{fancyhdr}
\usepackage{mathtools}

\DeclarePairedDelimiter\floor{\lfloor}{\rfloor}

\pagestyle{fancy}
\fancyhf{}
\hypersetup{ colorlinks=true, linkcolor=black, filecolor=magenta, urlcolor=cyan}
\geometry{ a4paper, total={170mm,257mm}, top=20mm, right=20mm, bottom=20mm, left=20mm}
\lhead{Pra GEMASTIK 14 ITB | Competitive Programming}
\setlength{\parindent}{0pt}
\setlength{\parskip}{0.3em}
\renewcommand{\headrulewidth}{0pt}
\rfoot{\thepage}
\lfoot{Pra GEMASTIK 14 ITB | Competitive Programming}
\lstset{
    basicstyle=\ttfamily\small,
    columns=fixed,
    extendedchars=true,
    breaklines=true,
    tabsize=2,
    prebreak=\raisebox{0ex}[0ex][0ex]{\ensuremath{\hookleftarrow}},
    frame=none,
    showtabs=false,
    showspaces=false,
    showstringspaces=false,
    prebreak={},
    keywordstyle=\color[rgb]{0.627,0.126,0.941},
    commentstyle=\color[rgb]{0.133,0.545,0.133},
    stringstyle=\color[rgb]{01,0,0},
    captionpos=t,
    escapeinside={(\%}{\%)}
}

\begin{document}

\begin{center}
    \section*{Problem E - EZ, lah!} % ganti judul soal

    \begin{tabular}{ | c c | }
        \hline
        Batas Waktu  & 2s \\    % jangan lupa ganti time limit
        Batas Memori & 256MB \\  % jangan lupa ganti memory limit
        \hline
    \end{tabular}
\end{center}

\subsection*{Deskripsi}

Anda dan teman anda terperangkap dalam sebuah ruangan. Untuk dapat terbebas dari ruangan tersebut, kalian perlu memecahkan sebuah soal yang tertulis di dalam ruangan tersebut sebagai berikut. Terdapat sebuah array $A$ yang memiliki $N$ buah elemen. Anda diminta untuk mencari jarak terdekat dari dua buah elemen pada $A$ yang nilainya terkecil. Dijamin terdapat setidaknya dua buah elemen pada $A$ yang nilainya merupakan nilai minimum dari elemen-elemen di $A$.

Teman anda berseru, \textbf{"EZ, lah!"}

\subsection*{Format Masukan}

Baris pertama terdiri dari satu bilangan bulat positif $N$ ($2 \leq N \leq 100.000$), menyatakan ukuran dari array $A$.

Baris kedua terdiri atas $N$ buah bilangan $A_1, A_2, \dots, A_N$ yang merupakan elemen-elemen pada array $A$, dipisahkan oleh spasi ($1 \leq A_i \leq 10^9$)

Dijamin setidaknya ada dua elemen pada $A$ yang bernilai $\min(A_1, A_2, \dots, A_N)$.

\subsection*{Format Keluaran}

Sebuah bilangan yang merupakan jawaban dari persoalan di atas
\\

\begin{multicols}{2}
\subsection*{Contoh Masukan}
\begin{lstlisting}
3
2 1 1
\end{lstlisting}
\columnbreak
\subsection*{Contoh Keluaran}
\begin{lstlisting}
1
\end{lstlisting}
\vfill
\null
\end{multicols}

\begin{multicols}{2}
\subsection*{Contoh Masukan}
\begin{lstlisting}
15
3 2 1 3 5 4 5 5 4 1 3 3 2 3 1
\end{lstlisting}
\columnbreak
\subsection*{Contoh Keluaran}
\begin{lstlisting}
5
\end{lstlisting}
\vfill
\null
\end{multicols}

% \subsection*{Penjelasan}
% Jika dibutuhkan, tambahkan penjelasan di sini

\pagebreak

\end{document}