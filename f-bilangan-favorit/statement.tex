\documentclass{article}

\usepackage{geometry}
\usepackage{amsmath}
\usepackage{graphicx, eso-pic}
\usepackage{listings}
\usepackage{hyperref}
\usepackage{multicol}
\usepackage{fancyhdr}
\usepackage{mathtools}

\DeclarePairedDelimiter\floor{\lfloor}{\rfloor}

\pagestyle{fancy}
\fancyhf{}
\hypersetup{ colorlinks=true, linkcolor=black, filecolor=magenta, urlcolor=cyan}
\geometry{ a4paper, total={170mm,257mm}, top=20mm, right=20mm, bottom=20mm, left=20mm}
\lhead{Pra GEMASTIK 14 ITB | Competitive Programming}
\setlength{\parindent}{0pt}
\setlength{\parskip}{0.3em}
\renewcommand{\headrulewidth}{0pt}
\rfoot{\thepage}
\lfoot{Pra GEMASTIK 14 ITB | Competitive Programming}
\lstset{
    basicstyle=\ttfamily\small,
    columns=fixed,
    extendedchars=true,
    breaklines=true,
    tabsize=2,
    prebreak=\raisebox{0ex}[0ex][0ex]{\ensuremath{\hookleftarrow}},
    frame=none,
    showtabs=false,
    showspaces=false,
    showstringspaces=false,
    prebreak={},
    keywordstyle=\color[rgb]{0.627,0.126,0.941},
    commentstyle=\color[rgb]{0.133,0.545,0.133},
    stringstyle=\color[rgb]{01,0,0},
    captionpos=t,
    escapeinside={(\%}{\%)}
}

\begin{document}

\begin{center}
    \section*{Problem F - Bilangan Favorit} % ganti judul soal

    \begin{tabular}{ | c c | }
        \hline
        Batas Waktu  & 2s \\    % jangan lupa ganti time limit
        Batas Memori & 256MB \\  % jangan lupa ganti memory limit
        \hline
    \end{tabular}
\end{center}

\subsection*{Deskripsi}

Teman anda punya dua buah digit favorit tak nol $X$ dan $Y$. Sebuah bilangan disebut sebagai bilangan \textit{kece} apabila bilangan tersebut hanya terdiri atas digit-digit $X$ dan $Y$. Bilangan favorit teman anda merupakan bilangan \textit{kece} yang jumlah dari digit-digitnya merupakan bilangan \textit{kece} juga. 

Anda memberikan sebuah persoalan kepadanya: berapa banyak bilangan favoritmu yang memiliki $N$ buah digit? Teman anda bingung. Bantu teman anda untuk menjawab pertanyaan anda sendiri!

Karena jawaban bisa bernilai besar, keluarkan dalam modulo $10^9 + 7$.

\subsection*{Format Masukan}

Sebuah baris yang terdiri atas tiga bilangan bulat positif $X$, $Y$, dan $N$ yang dipisahkan oleh spasi ($1 \leq X, Y \leq 9$, $1 \leq N \leq 10^6$).

\subsection*{Format Keluaran}

Sebuah bilangan yang merupakan banyaknya bilangan favorit teman anda yang terdiri atas $N$ digit.
\\

\begin{multicols}{2}
\subsection*{Contoh Masukan}
\begin{lstlisting}
1 3 5
\end{lstlisting}
\columnbreak
\subsection*{Contoh Keluaran}
\begin{lstlisting}
15
\end{lstlisting}
\vfill
\null
\end{multicols}

\begin{multicols}{2}
\subsection*{Contoh Masukan}
\begin{lstlisting}
1 8 2
\end{lstlisting}
\columnbreak
\subsection*{Contoh Keluaran}
\begin{lstlisting}
0
\end{lstlisting}
\vfill
\null
\end{multicols}

\subsection*{Penjelasan}
% Jika dibutuhkan, tambahkan penjelasan di sini
Pada contoh pertama, seluruh bilangan favorit teman anda adalah: 

\begin{multicols}{3}
\begin{itemize}
    \item $11333$
    \item $13133$
    \item $13313$
    \item $13331$
    \item $13333$
\end{itemize}
\columnbreak
\begin{itemize}
    \item $31133$
    \item $31313$
    \item $31331$
    \item $31333$
    \item $33113$
\end{itemize}

\columnbreak
\begin{itemize}
    \item $33131$
    \item $33133$
    \item $33311$
    \item $33313$
    \item $33331$
\end{itemize}
\columnbreak
\end{multicols}

\pagebreak
\end{document}