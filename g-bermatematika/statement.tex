\documentclass{article}

\usepackage{geometry}
\usepackage{amsmath}
\usepackage{graphicx, eso-pic}
\usepackage{listings}
\usepackage{hyperref}
\usepackage{multicol}
\usepackage{fancyhdr}
\usepackage{mathtools}

\DeclarePairedDelimiter\floor{\lfloor}{\rfloor}

\pagestyle{fancy}
\fancyhf{}
\hypersetup{ colorlinks=true, linkcolor=black, filecolor=magenta, urlcolor=cyan}
\geometry{ a4paper, total={170mm,257mm}, top=20mm, right=20mm, bottom=20mm, left=20mm}
\lhead{Pra GEMASTIK 14 ITB | Competitive Programming}
\setlength{\parindent}{0pt}
\setlength{\parskip}{0.3em}
\renewcommand{\headrulewidth}{0pt}
\rfoot{\thepage}
\lfoot{Pra GEMASTIK 14 ITB | Competitive Programming}
\lstset{
    basicstyle=\ttfamily\small,
    columns=fixed,
    extendedchars=true,
    breaklines=true,
    tabsize=2,
    prebreak=\raisebox{0ex}[0ex][0ex]{\ensuremath{\hookleftarrow}},
    frame=none,
    showtabs=false,
    showspaces=false,
    showstringspaces=false,
    prebreak={},
    keywordstyle=\color[rgb]{0.627,0.126,0.941},
    commentstyle=\color[rgb]{0.133,0.545,0.133},
    stringstyle=\color[rgb]{01,0,0},
    captionpos=t,
    escapeinside={(\%}{\%)}
}

\begin{document}

\begin{center}
    \section*{Problem G - Bermatematika} % ganti judul soal

    \begin{tabular}{ | c c | }
        \hline
        Batas Waktu  & 1s \\    % jangan lupa ganti time limit
        Batas Memori & 256MB \\  % jangan lupa ganti memory limit
        \hline
    \end{tabular}
\end{center}

\subsection*{Deskripsi}

Anda sedang menonton siaran dari channel Bermatematika.com di YouTube. Pada siaran tersebut diberikan sebuah \textit{challenge} berhadiah. Diberikan dua buah bilangan bulat $X$ dan $Y$, penonton diminta untuk mencari sebuah bilangan bulat non-negatif $K$ sehingga KPK (Kelipatan Persekutuan Terkecil) dari $X+K$ dan $Y+K$ bernilai sekecil mungkin. Apabila ada beberapa nilai $K$ yang memenuhi, nilai terkecil lah yang menjadi jawabannya. "Ini mah berinformatika," gerutu anda dalam hati. Karena ingin hadiah, anda segera membuka editor kesayangan anda, dan mulai melakukan \textit{coding}.

\subsection*{Format Masukan}

Sebuah baris yang terdiri atas dua bilangan bulat positif $X$ dan $Y$ ($1 \leq X, Y \leq 10^9$).

\subsection*{Format Keluaran}

Sebuah baris yang menyatakan jawaban dari \textit{challenge} yang diberikan oleh channel Bermatematika.com. Jawaban dijamin muat dalam Integer 64 bit.
\\

\begin{multicols}{2}
\subsection*{Contoh Masukan}
\begin{lstlisting}
6 10
\end{lstlisting}
\columnbreak
\subsection*{Contoh Keluaran}
\begin{lstlisting}
2
\end{lstlisting}
\vfill
\null
\end{multicols}

\subsection*{Penjelasan}
% Jika dibutuhkan, tambahkan penjelasan di sini
Nilai $KPK(6 + 2, 10 + 2)$ adalah $24$. Dapat ditunjukkan bahwa $24$ merupakan yang terkecil dari seluruh nilai $KPK(6 + K, 10 + K)$ untuk setiap bilangan bulat non-negatif $K$, dan $K$ terkecil yang memenuhi adalah $2$.

\pagebreak

\end{document}