\documentclass{article}

\usepackage{geometry}
\usepackage{amsmath}
\usepackage{graphicx, eso-pic}
\usepackage{listings}
\usepackage{hyperref}
\usepackage{multicol}
\usepackage{fancyhdr}
\usepackage{mathtools}

\DeclarePairedDelimiter\floor{\lfloor}{\rfloor}

\pagestyle{fancy}
\fancyhf{}
\hypersetup{ colorlinks=true, linkcolor=black, filecolor=magenta, urlcolor=cyan}
\geometry{ a4paper, total={170mm,257mm}, top=20mm, right=20mm, bottom=20mm, left=20mm}
\lhead{Pra GEMASTIK 14 ITB | Competitive Programming}
\setlength{\parindent}{0pt}
\setlength{\parskip}{0.3em}
\renewcommand{\headrulewidth}{0pt}
\rfoot{\thepage}
\lfoot{Pra GEMASTIK 14 ITB | Competitive Programming}
\lstset{
    basicstyle=\ttfamily\small,
    columns=fixed,
    extendedchars=true,
    breaklines=true,
    tabsize=2,
    prebreak=\raisebox{0ex}[0ex][0ex]{\ensuremath{\hookleftarrow}},
    frame=none,
    showtabs=false,
    showspaces=false,
    showstringspaces=false,
    prebreak={},
    keywordstyle=\color[rgb]{0.627,0.126,0.941},
    commentstyle=\color[rgb]{0.133,0.545,0.133},
    stringstyle=\color[rgb]{01,0,0},
    captionpos=t,
    escapeinside={(\%}{\%)}
}

\begin{document}

\begin{center}
    \section*{Problem H - Eksplorasi Array} % ganti judul soal

    \begin{tabular}{ | c c | }
        \hline
        Batas Waktu  & 2s \\    % jangan lupa ganti time limit
        Batas Memori & 256MB \\  % jangan lupa ganti memory limit
        \hline
    \end{tabular}
\end{center}

\subsection*{Deskripsi}

Terdapat dua buah array $X = [X_1, X_2, \dots, X_N]$ dan $Y = [Y_1, Y_2, \dots, Y_N]$.
Anda diperbolehkan melakukan eksplorasi pada array $X$. Langkah yang diperbolehkan dalam melakukan eksplorasi adalah sebagai berikut:

\begin{itemize}
    \item Pilih dua buah bilangan $i, j$ ($1 \le i \le j \le N$).
    \item Lakukan sorting pada sub-array $X$ yang dimulai dari indeks ke-$i$ hingga indeks ke-$j$.
\end{itemize}

Sebagai contoh, apabila pada suatu saat $X = [1, 3, 7, 2, 9]$ dan $i = 2, j = 4$, maka setelah langkah tersebut $X = [1, 2, 3, 7, 9]$.

Langkah dapat dilakukan sebanyak apapun. Anda sekarang bertanya-tanya apabila setelah beberapa kali melakukan langkah eksplorasi, array $X$ dapat menjadi array $Y$. Temukan jawabannya!

\subsection*{Format Masukan}

% Baris pertama terdiri dari satu bilangan bulat positif $T$ ($1 \leq T \leq 3 \times 10^5$), menyatakan banyaknya kasus uji.

% Baris pertama tiap kasus uji diawali dengan bilangan $N$ ($1 \leq N \leq 10^5$).

% Baris kedua tiap kasus uji terdiri atas $N$ buah bilangan $X_1, \dots X_N$ ($1 \leq X_i \leq N$).

% Baris kedua tiap kasus uji terdiri atas $N$ buah bilangan $Y_1, \dots Y_N$ ($1 \leq Y_i \leq N$).

% Jumlah $N$ pada seluruh kasus uji dijamin bernilai tidak lebih dari $3 \times 10^5$.

Baris pertama diawali dengan bilangan $N$ ($1 \leq N \leq 10^5$).

Baris kedua terdiri atas $N$ buah bilangan $X_1, \dots X_N$ ($1 \leq X_i \leq N$).

Baris ketiga terdiri atas $N$ buah bilangan $Y_1, \dots Y_N$ ($1 \leq Y_i \leq N$).

\subsection*{Format Keluaran}

Sebuah string yang bertuliskan \lstinline{Sabi} jika $X$ bisa menjadi $Y$, otherwise \lstinline{Gasabi}.
\\

\begin{multicols}{2}
\subsection*{Contoh Masukan}
\begin{lstlisting}
8
2 3 4 5 6 5 1 4
1 2 3 4 5 6 5 4
\end{lstlisting}
\columnbreak
\subsection*{Contoh Keluaran}
\begin{lstlisting}
Sabi
\end{lstlisting}
\vfill
\null
\end{multicols}

% \subsection*{Penjelasan}
% Jika dibutuhkan, tambahkan penjelasan di sini

\pagebreak

\end{document}