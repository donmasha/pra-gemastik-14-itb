\documentclass{article}

\usepackage{geometry}
\usepackage{amsmath}
\usepackage{graphicx, eso-pic}
\usepackage{listings}
\usepackage{hyperref}
\usepackage{multicol}
\usepackage{fancyhdr}
\usepackage{mathtools}

\DeclarePairedDelimiter\floor{\lfloor}{\rfloor}

\pagestyle{fancy}
\fancyhf{}
\hypersetup{ colorlinks=true, linkcolor=black, filecolor=magenta, urlcolor=cyan}
\geometry{ a4paper, total={170mm,257mm}, top=20mm, right=20mm, bottom=20mm, left=20mm}
\lhead{Pra GEMASTIK 14 ITB | Competitive Programming}
\setlength{\parindent}{0pt}
\setlength{\parskip}{0.3em}
\renewcommand{\headrulewidth}{0pt}
\rfoot{\thepage}
\lfoot{Pra GEMASTIK 14 ITB | Competitive Programming}
\lstset{
    basicstyle=\ttfamily\small,
    columns=fixed,
    extendedchars=true,
    breaklines=true,
    tabsize=2,
    prebreak=\raisebox{0ex}[0ex][0ex]{\ensuremath{\hookleftarrow}},
    frame=none,
    showtabs=false,
    showspaces=false,
    showstringspaces=false,
    prebreak={},
    keywordstyle=\color[rgb]{0.627,0.126,0.941},
    commentstyle=\color[rgb]{0.133,0.545,0.133},
    stringstyle=\color[rgb]{01,0,0},
    captionpos=t,
    escapeinside={(\%}{\%)}
}

\begin{document}

\begin{center}
    \section*{Problem J - Bermain Game} % ganti judul soal

    \begin{tabular}{ | c c | }
        \hline
        Batas Waktu  & 1s \\    % jangan lupa ganti time limit
        Batas Memori & 128MB \\  % jangan lupa ganti memory limit
        \hline
    \end{tabular}
\end{center}

\subsection*{Deskripsi}

Anda dan teman anda sedang bermain. Permainannya adalah sebagai berikut. Diberikan sebuah bilangan bulat positif $N > 1$ dan awalnya bilangan $1$ tertulis pada papan tulis.

Secara bergantian kalian akan melangkah, dimulai dari Anda. Pada setiap langkah, apabila pada saat itu bilangan pada papan tulis adalah $X$, maka pemain dapat mengganti bilangan tersebut dengan $X + 1$ atau $2X$, tetapi bilangan baru tersebut tidak boleh bernilai lebih dari $N$. Pemenangnya adalah pemain yang menuliskan bilangan $N$ pada papan tulis.

Sekarang Anda bertanya-tanya, apabila diberikan dua buah bilangan $L$ dan $R$ ($2 \le L \le R$), ada berapa banyak bilangan di antara $L$ dan $R$ (inklusif) yang dapat menjadi nilai $N$ sehingga anda memiliki strategi untuk dapat memenangkan permainan tersebut dengan pasti. Diasumsikan Anda dan teman anda bermain dengan strategi yang optimal.

\subsection*{Format Masukan}

Baris pertama terdiri dari satu bilangan bulat positif $T$ ($1 \leq T \leq 10^4$), menyatakan banyaknya kasus uji.
Tiap kasus uji berisi dua bilangan bulat positif $L$ dan $R$ ($2 \leq L \leq R \leq 10^{18}$).

\subsection*{Format Keluaran}

Keluarkan banyaknya bilangan bulat pada interval $[L, R]$ yang dapat menjadi nilai $N$ sehingga Anda pasti menang.

\begin{multicols}{2}
\subsection*{Contoh Masukan}
\begin{lstlisting}
2
2 3
3 5
\end{lstlisting}
\columnbreak
\subsection*{Contoh Keluaran}
\begin{lstlisting}
1
0
\end{lstlisting}
\vfill
\null
\end{multicols}

\subsection*{Penjelasan}
Pada contoh kasus uji pertama, hanya $N = 2$ yang membuat Anda menang.

Pada contoh kasus uji kedua, tidak ada nilai $N$ pada $\left\{3,4,5\right\}$ yang dapat membuat Anda menang.
\pagebreak

\end{document}