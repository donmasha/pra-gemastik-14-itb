\documentclass{article}

\usepackage{geometry}
\usepackage{amsmath}
\usepackage{graphicx, eso-pic}
\usepackage{listings}
\usepackage{hyperref}
\usepackage{multicol}
\usepackage{fancyhdr}
\usepackage{mathtools}

\DeclarePairedDelimiter\floor{\lfloor}{\rfloor}

\pagestyle{fancy}
\fancyhf{}
\hypersetup{ colorlinks=true, linkcolor=black, filecolor=magenta, urlcolor=cyan}
\geometry{ a4paper, total={170mm,257mm}, top=20mm, right=20mm, bottom=20mm, left=20mm}
\lhead{Pra GEMASTIK 14 ITB | Competitive Programming}
\setlength{\parindent}{0pt}
\setlength{\parskip}{0.3em}
\renewcommand{\headrulewidth}{0pt}
\rfoot{\thepage}
\lfoot{Pra GEMASTIK 14 ITB | Competitive Programming}
\lstset{
    basicstyle=\ttfamily\small,
    columns=fixed,
    extendedchars=true,
    breaklines=true,
    tabsize=2,
    prebreak=\raisebox{0ex}[0ex][0ex]{\ensuremath{\hookleftarrow}},
    frame=none,
    showtabs=false,
    showspaces=false,
    showstringspaces=false,
    prebreak={},
    keywordstyle=\color[rgb]{0.627,0.126,0.941},
    commentstyle=\color[rgb]{0.133,0.545,0.133},
    stringstyle=\color[rgb]{01,0,0},
    captionpos=t,
    escapeinside={(\%}{\%)}
}

\begin{document}

\begin{center}
    \section*{Problem I - Berpetualang} % ganti judul soal

    \begin{tabular}{ | c c | }
        \hline
        Batas Waktu  & 2s \\    % jangan lupa ganti time limit
        Batas Memori & 256MB \\  % jangan lupa ganti memory limit
        \hline
    \end{tabular}
\end{center}

\subsection*{Deskripsi}

Master prime sedang berpetualang di suatu tempat. Pada tempat tersebut ada $N$ pulau yang dinomori dari 1 sampai $N$, pulau tersebut terhubung dengan $M$ jembatan dua-arah.

Dipastikan bahwa selalu bisa pergi dari suatu pulau ke pulau yang lain. 

Master prime suka berpetualang, jika dia berpetualang ke suatu tempat, maka akan dicatat ke bukunya, dan jika sudah ada pulaunya di bukunya maka tidak akan dicatat ulang.

Master prime sedang ada di pulau nomor 1 dan ingin berpetualang ke semua pulau, carilah cara agar master prime mendapatkan \textbf{urutan pulau yang} \href{https://en.wikipedia.org/wiki/Lexicographic_order}{lexicographically} \textbf{minimum} tercatat di bukunya!

\subsection*{Format Masukan}

Baris pertama berisi dua buah bilanga bulat $N, M$ $(1 \leq N, M \leq 10^5)$, menyatakan banyak pulau dan banyak jembatan yang ada.

$M$ baris berikutnya berisi dua bilangan bulat $U_i, V_i$ menyatakan bahwa pulau $U_i$ dan pulau $V_i$ terhubung dengan jembatan dua-arah. $(1 \leq U_i, V_i \leq N)$

\textbf{Catatan:}

Bisa saja ada lebih dari satu jembatan antar kota, dan bisa saja ada jembatan yang menghubungi ke kota itu sendiri.

\subsection*{Format Keluaran}

Keluarkan $N$ bilangan bulat $P_1, \dots, P_N$ berupa jawaban dari persoalan ini!
\\

\begin{multicols}{2}
\subsection*{Contoh Masukan}
\begin{lstlisting}
5 5
1 4
3 4
5 4
3 2
1 5
\end{lstlisting}
\columnbreak
\subsection*{Contoh Keluaran}
\begin{lstlisting}
1 4 3 2 5 
\end{lstlisting}
\vfill
\null
\end{multicols}

\subsection*{Penjelasan}

Master prima dapat pergi dengan berpetualang sebagai berikut:

$$1 \rightarrow 4 \rightarrow 3 \rightarrow 2 \rightarrow 3 \rightarrow 4 \rightarrow 1 \rightarrow 5$$

Sehingga didapat urutan 1, 4, 3, 2, 5.

\pagebreak

\end{document}