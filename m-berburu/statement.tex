\documentclass{article}

\usepackage{geometry}
\usepackage{amsmath}
\usepackage{graphicx, eso-pic}
\usepackage{listings}
\usepackage{hyperref}
\usepackage{multicol}
\usepackage{fancyhdr}
\usepackage{mathtools}

\DeclarePairedDelimiter\floor{\lfloor}{\rfloor}

\pagestyle{fancy}
\fancyhf{}
\hypersetup{ colorlinks=true, linkcolor=black, filecolor=magenta, urlcolor=cyan}
\geometry{ a4paper, total={170mm,257mm}, top=20mm, right=20mm, bottom=20mm, left=20mm}
\lhead{Pra GEMASTIK 14 ITB | Competitive Programming}
\setlength{\parindent}{0pt}
\setlength{\parskip}{0.3em}
\renewcommand{\headrulewidth}{0pt}
\rfoot{\thepage}
\lfoot{Pra GEMASTIK 14 ITB | Competitive Programming}
\lstset{
    basicstyle=\ttfamily\small,
    columns=fixed,
    extendedchars=true,
    breaklines=true,
    tabsize=2,
    prebreak=\raisebox{0ex}[0ex][0ex]{\ensuremath{\hookleftarrow}},
    frame=none,
    showtabs=false,
    showspaces=false,
    showstringspaces=false,
    prebreak={},
    keywordstyle=\color[rgb]{0.627,0.126,0.941},
    commentstyle=\color[rgb]{0.133,0.545,0.133},
    stringstyle=\color[rgb]{01,0,0},
    captionpos=t,
    escapeinside={(\%}{\%)}
}

\begin{document}

\begin{center}
    \section*{Problem M - Kucing Lieur} % ganti judul soal

    \begin{tabular}{ | c c | }
        \hline
        Batas Waktu  & 2s \\    % jangan lupa ganti time limit
        Batas Memori & 256MB \\  % jangan lupa ganti memory limit
        \hline
    \end{tabular}
\end{center}

\subsection*{Deskripsi}

Terdapat suatu Asrama IAIC (Ikatan Alumni Ingin Cat) yang terdiri dari $N$ kamar dan satu kucing lieur!

Orang-orang di Asrama IAIC memutuskan untuk memasang perangkap di beberapa ruangan untuk mengambil kucing lieur yang terkenal lucu itu.

Memasang jebakan di kamar nomor $X$ membutuhkan biaya $P_X$ rupiah. Kamar diberi nomor dari 1 sampai $N$.

Kucing lieur tidak duduk di tempat sepanjang waktu, ia terus berjalan. Jika berada di kamar $X$ di detik $T$ maka dia akan lari ke kamar $A_X$ di detik $T + 1$ tanpa mengunjungi kamar lain diantaranya ($X = A_X$ berarti kucing tidak akan meninggalkan kamar $X$).

Pemasangan perangkap ini terjadi pada detik 0. Jika kucing berada di suatu ruangan dengan perangkap di dalamnya, maka kucing tersebut akan terjebak ke dalam perangkap tersebut.

Itu akan sangat mudah jika para orang di asrama itu benar-benar tahu di mana kucing itu berada. Sayangnya, bukan itu masalahnya, kucing dapat berada di ruangan mana pun dari 1 hingga $N$ pada detik 0.

Master prime sebagai ketua IAIC bertanya kepada Anda. Berapa jumlah minimal rupiah yang dapat dihabiskan untuk memasang perangkap untuk menjamin bahwa kucing pada akhirnya akan ditangkap tidak peduli dari ruangan mana kucing itu berasal?

Bantulah Master prime ini!

\subsection*{Format Masukan}

Baris pertama berisi satu bilangan bulat $N$ $(1 \leq N \leq 2 \times 10^5)$ — jumlah kamar di asrama.

Baris kedua berisi $N$ bilangan bulat $P_1,P_2,\dots,P_N$ $(1 \leq P_i \leq 10^4)$ — $P_i$ adalah biaya pemasangan perangkap di kamar nomor $i$.

Baris ketiga berisi $N$ bilangan bulat $A_1,A_2,…,A_N$ $(1 \leq A_i \leq N)$ — $A_i$ adalah ruangan yang akan dijalankan kucing ke detik berikutnya setelah berada di ruangan $i$.

\subsection*{Format Keluaran}

Keluarkan satu bilangan bulat, berupa jawaban dari persoalan ini!
\\

\begin{multicols}{2}
\subsection*{Contoh Masukan 1}
\begin{lstlisting}
10
6 9 1 1 1 10 2 4 9 6
5 3 4 2 6 8 9 1 10 7
\end{lstlisting}
\columnbreak
\subsection*{Contoh Keluaran 1}
\begin{lstlisting}
4
\end{lstlisting}
\vfill
\null
\end{multicols}

\begin{multicols}{2}
\subsection*{Contoh Masukan 2}
\begin{lstlisting}
10
9 19 19 1 3 9 1 12 10 8
7 1 6 3 4 8 5 10 2 9
\end{lstlisting}
\columnbreak
\subsection*{Contoh Keluaran 2}
\begin{lstlisting}
1
\end{lstlisting}
\vfill
\null
\end{multicols}

% \subsection*{Penjelasan}
% Jika dibutuhkan, tambahkan penjelasan di sini

\pagebreak

\end{document}