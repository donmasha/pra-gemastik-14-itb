\documentclass{article}

\usepackage{geometry}
\usepackage{amsmath}
\usepackage{graphicx, eso-pic}
\usepackage{listings}
\usepackage{hyperref}
\usepackage{multicol}
\usepackage{fancyhdr}
\usepackage{mathtools}

\DeclarePairedDelimiter\floor{\lfloor}{\rfloor}

\pagestyle{fancy}
\fancyhf{}
\hypersetup{ colorlinks=true, linkcolor=black, filecolor=magenta, urlcolor=cyan}
\geometry{ a4paper, total={170mm,257mm}, top=20mm, right=20mm, bottom=20mm, left=20mm}
\lhead{Pra GEMASTIK 14 ITB | Competitive Programming}
\setlength{\parindent}{0pt}
\setlength{\parskip}{0.3em}
\renewcommand{\headrulewidth}{0pt}
\rfoot{\thepage}
\lfoot{Pra GEMASTIK 14 ITB | Competitive Programming}
\lstset{
    basicstyle=\ttfamily\small,
    columns=fixed,
    extendedchars=true,
    breaklines=true,
    tabsize=2,
    prebreak=\raisebox{0ex}[0ex][0ex]{\ensuremath{\hookleftarrow}},
    frame=none,
    showtabs=false,
    showspaces=false,
    showstringspaces=false,
    prebreak={},
    keywordstyle=\color[rgb]{0.627,0.126,0.941},
    commentstyle=\color[rgb]{0.133,0.545,0.133},
    stringstyle=\color[rgb]{01,0,0},
    captionpos=t,
    escapeinside={(\%}{\%)}
}

\begin{document}

\begin{center}
    \section*{Problem N - Mencari Frekuensi} % ganti judul soal

    \begin{tabular}{ | c c | }
        \hline
        Batas Waktu  & 2s \\    % jangan lupa ganti time limit
        Batas Memori & 256MB \\  % jangan lupa ganti memory limit
        \hline
    \end{tabular}
\end{center}

\subsection*{Deskripsi}

Diberikan $N$ bilangan bulat $A_1, A_2, \dots, A_N$, dan $Q$ query, dengan setiap query berisi empat bilangan bulat $L, R, P, Q$ $(L \leq R, P \leq Q)$. Pada setiap query tersebut, Anda disuruh mencari banyaknya bilangan bulat $X$ yang nilai frekuensinya diantara $P$ dan $Q$ (inklusi) yang berada pada subarray $[L, R]$ $(A_L, A_{L + 1}, \dots, A_R)$

\subsection*{Format Masukan}

Baris pertama berisi dua bilangan bulat yang dipisahkan spasi, $N$ dan $Q$ $(1 \leq N, Q \leq 5 \times 10^4)$.

Baris kedua berisi $N$ bilangan bulat yang dipisahkan oleh spasi, yakni nilai dari $A_1, A_2, \dots, A_N$ $(1 \leq A_i \leq 5 \times 10^4)$.

$Q$ baris berikutnya mewakili query.

Setiap query berisi 4 bilangan bulat $L, R, P, Q$  $(1 \leq L \leq R \leq N, 1 \leq P \leq Q \leq N)$.

\subsection*{Format Keluaran}

Untuk tiap query, tuliskan jawaban dari query tersebut!

\begin{multicols}{2}
\subsection*{Contoh Masukan}
\begin{lstlisting}
10
1 3 5 1 8 0 1 2 1 2
10
1 3 1 1
1 3 1 3
1 7 3 3
1 7 2 2
5 10 1 2
5 10 2 2
1 10 1 1
1 10 2 2
1 10 3 3
1 10 1 3
\end{lstlisting}
\columnbreak
\subsection*{Contoh Keluaran}
\begin{lstlisting}
3
3
1
0
4
2
4
1
0
5
\end{lstlisting}
\vfill
\null
\end{multicols}

% \subsection*{Penjelasan}
% Jika dibutuhkan, tambahkan penjelasan di sini

\pagebreak

\end{document}