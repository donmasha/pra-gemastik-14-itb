\documentclass{article}

\usepackage{geometry}
\usepackage{amsmath}
\usepackage{graphicx}
\usepackage{listings}
\usepackage{hyperref}
\usepackage{multicol}
\usepackage{fancyhdr}
\pagestyle{fancy}
\hypersetup{ colorlinks=true, linkcolor=black, filecolor=magenta, urlcolor=cyan}
\geometry{ a4paper, total={170mm,257mm}, top=20mm, right=20mm, bottom=20mm, left=20mm}
\setlength{\parindent}{0pt}
\setlength{\parskip}{1em}
\renewcommand{\headrulewidth}{0pt}
\lhead{ITB - IEEEXtreme 14.0 Selection}
\fancyfoot[CE,CO]{\thepage}
\lstset{
    basicstyle=\ttfamily\small,
    columns=fixed,
    extendedchars=true,
    breaklines=true,
    tabsize=2,
    prebreak=\raisebox{0ex}[0ex][0ex]{\ensuremath{\hookleftarrow}},
    frame=none,
    showtabs=false,
    showspaces=false,
    showstringspaces=false,
    prebreak={},
    keywordstyle=\color[rgb]{0.627,0.126,0.941},
    commentstyle=\color[rgb]{0.133,0.545,0.133},
    stringstyle=\color[rgb]{01,0,0},
    captionpos=t,
    escapeinside={(\%}{\%)}
}

\begin{document}

\begin{center}
    \section*{I. Menghitung Buku}

    \begin{tabular}{ | c c | }
        \hline
        Batas Waktu  & 1s \\    % jangan lupa ganti time limit
        Batas Memori & 256MB \\  % jangan lupa ganti memory limit
        \hline
    \end{tabular}
\end{center}

\subsection*{Deskripsi}

Hasan adalah seorang mahasiswa ITB yang sangat rajin. Dia memiliki buku yang bernomor $1, 2, 3, \dots$ sampai tak terhingga. Kamal, temannya Hasan, mengambil semua buku - buku yang memiliki nomor berkelipatan $N$. Sekarang Hasan penasaran, ia ingin menghitung apa nomor yang ada di buku pada urutan ke $K$.

Misalnya, jika $N$ bernilai $4$ dan $K$ bernilai $7$, maka barisannya adalah $1, 2, 3, 5, 6, 7, 9, 10, 11, \dots$ dan buku yang berada pada urutan ke $7$ adalah buku bernomor $9$.

\subsection*{Format Masukan}

Baris pertama terdiri dari satu bilangan bulat positif $T$ ($1 \leq T \leq 1000$), menyatakan banyaknya kasus uji.

$T$ Baris berikutnya berisi $2$ buah bilangan $N$ ($2 \leq N \leq 10^9$) dan $K$ ($2 \leq K \leq 10^9$).

\subsection*{Format Keluaran}

Keluarkan $T$ buah baris yang berisi jawaban untuk setiap kasus uji.
\\

\begin{multicols}{2}
\subsection*{Contoh Masukan}
\begin{lstlisting}
5
7 4
4 7
2 100
5 12
500000000 1000000000
\end{lstlisting}
\columnbreak
\subsection*{Contoh Keluaran}
\begin{lstlisting}
4
9
199
14
1000000002
\end{lstlisting}
\vfill
\null
\end{multicols}

\subsection*{Penjelasan}
Untuk kasus pertama, barisannya adalah $1, 2, 3, 4, 5, 6, 8, 9, \dots$. Dapat dilihat bahwa buku ke-$4$ adalah buku bernomor $4$.

Kasus kedua telah dijelaskan di deskripsi soal.

\pagebreak

\end{document}